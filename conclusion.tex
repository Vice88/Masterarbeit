\chapter{Conclusions and outlook}
% Verbesserungsvorschlag, was muss getan werden um einen Demonstrator auszuliefern?
% Aussagekräftige Messung durchführen, evtl. Bonddrähte neu setzen!
% Alles andere kann nicht mehr geändert werden. 
%\textit{A summary of the most important results, whereby a repeated emphasis of their relevance, importance and novelty cannot hurt. A brief precis of the envisaged future potential of the work is suitable here, but avoid addressing the Nobel Committee directly.}
% CMOS driver + GaN power stage cite(A Package-Integrated 50W High-Efficiency RF CMOS-GaN Class-E
%Power Amplifier)
In this work the implementation of the Riemann Pump in a system design has been presented.
Further, the design process from a charge pump to a custom \gls{ab:dac} has been described.
In contrast to classical current steering topologies, this concept used the benefit of differential coding, which saves energy and exhibits a better performance.
A concise evaluation states that the Riemann Pump is a great improvement for conventional digital-to-analog conversion concepts.\\
The major problem in the design process was, that the current sources could not be defined as necessary, since the transistors have to operate in their saturation region, which could not be established for all signals.
As simulation showed, this problem occurred for both, the high and the low side switching transistor.
It is demonstrated that the relative slopes do not scale with the presented theoretical slopes.
One requirement to guarantee the proper scaling would be that the output voltage only vary between $V_{ss} + V_{ds,sat}$ and $V_{dd} - V_{ds,sat}$.
Non ideal switching occurs, which made it necessary to average the current over time.
The problem of not perfect switching is, that the channel is opened and closed slowly in comparison to an ideal switch.
Further, a leakage current over the driver circuit is observed.
The different behaviour for the switching process of the separated push-pull stages led to different slopes.
As no complementary transistors were available in III-V technology, a proper driver circuit had to be investigated.
Therefore a known concept \cite{MaksimovicPaper} for the driver circuit was implemented.
The dimension of the used components were calibrated with respect to the resulting voltage step at an output capacitance, which represent the input stage of a power amplifier.
The dimension of the used components was a trade-off between bandwidth limitation and power consumption.
Small transistor dimensions could synthesize signals with low frequencies and big dimensions with high frequencies.
For the chosen configuration of the presented concept, a signal bandwidth of roughly \SI{1.5}{\GHz} to \SI{6}{\GHz} was found.
To increase the baseband signal frequency, the dimension of the transistors had to be bigger but then they dissipate more power. \\
Simulations confirmed the feasibility to synthesize different waveforms, for example a sine wave, a rectified sine and a triangular waveform.
Setting the \gls{ab:osr} to four led to a sampling frequency \gls{sy:fsampling} which is eight times \gls{sy:fsignal}.
Therefore the unity gain frequency of the transistors $\approx f_{t} = 30 GHz$ had to be considered to fulfil this requirement.
For an increase of the \gls{ab:osr}, the signal quality is improved but the power consumption also scales linear with it.\\
Further simulations yielded, that the parasitic and loss effects of the driver circuit distorted the waveforms.
The highest \gls{ab:snr} (\SI{32.5}{\decibel}) found, was for a sine wave at \SI{6}{\giga \hertz} with an amplitude of $\hat{v} = 1.75 V$, an DC-offset of $V_{DC} = 13 V$ and a phase shift of $\phi = -\pi/8$, see appendix \ref{app:snr}.\\
In the realisation and layout process a few things had to be considered.
Important for the design of the input lines were, that they are of the same length, due to timing issues.
The same applies to the bond wires of the in- and output lines.
The input timing is crucial due to the fact that the switches have to switch synchronous.
One of the most important things was the dissipation of heat.
Two concepts were presented to dissipate the heat in the most proper way, which were fabricated by CONTAG AG.
An improvement regarding the layout, could be to solder the chips on an electrical insulator while thermal good conductor, as \gls{ab:aln}.
In the first measurement ever performed with a Riemann Pump in \gls{ab:gan} technology, the stability was checked.
This analysis proved that the system is stable and validated that the circuit do not oscillate.
It is stated, that the energy consumption is critical and is rather in the range for base stations than for mobile devices.
As the signal quality is the main property to consider, it is stated how to improve this quality.
To increase the signal quality, the resolution could be increased, but the trade-off is that the whole circuit would become more complex and thus the energy consumption would increase.
It is to mention, that the baseband signal can be shifted to even higher frequencies than the presented \SI{6}{\giga \hertz} by adapting the dimension of the used components, but the bandwidth stays constant while the power consumption increases.\\
In a next measurement, the proper switching of the push pull stage were confirmed.
After that, it was shown that different output signals could be synthesized.
In fact, it was shown that four different currents could be established, which represent the theoretical achievable four slopes of a two bit resolution.
It was possible to prove the demonstrated concept and show the feasibility of the designed demonstrator to synthesize signals with different amplitudes and frequencies.
The signal integrity suffered from measurement errors, non ideal switching and leakage currents.
Therefore the upper bound on the frequency range for the presented measurement setup was at roughly \SI{150}{\mega \hertz}.
In addition to the undesired effects of the circuit topology, the problematic of the heat dissipation came into play.
Hence the signal integrity was reduced by the degradation induced by the heat.
Nevertheless, the results demonstrate for the first time ever, the feasibility to convert digital signals into analog ones with this concept and technology.
It was the first time ever, building a concrete test circuit in \gls{ab:gan} technology, which provides decent measurement results.
Therefore it is to state, that the proof of concept was successfully demonstrated with the built test circuit.\\
A design and the processing of a \gls{ab:mmic} structure, containing the Riemann Pump, was beyond the scope of this thesis.
It is to mention, that the calculation of the Riemann Code have to be done with an external signal processor, which has to compute this code in real time. 
The real time requirement is because of a wanted latency of \SI{1}{\milli \second} for the communication link.
With the ability to adapt some system parameter, like the reference current and the oversampling ratio, it is possible to create arbitrary waveforms.\\
The next steps would be to implement an algorithm which computes the Riemann Code.
A MATLAB algorithm would compute this code by minimizing the deviation between a theoretical signal and the synthesized signal.
Then the used chips could be optimized regarding an efficient driver circuit.
This would be, to process a \gls{ab:mmic} which reduces the leakage effects.
Further, the switching transistors had to operate in their saturation region, which should be investigated in future researches.
After optimizing the circuits aspects, some assembly aspect have to be considered.
The implementation of a thermal conductor but electrical insulator for the high side switches, would be an improvement regarding the heat dissipation.



