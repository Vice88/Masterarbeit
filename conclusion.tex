\chapter{Conclusions and outlook}
% Verbesserungsvorschlag, was muss getan werden um einen Demonstrator auszuliefern?
% Aussagekräftige Messung durchführen, evtl. Bonddrähte neu setzen!
% Alles andere kann nicht mehr geändert werden. 
%\textit{A summary of the most important results, whereby a repeated emphasis of their relevance, importance and novelty cannot hurt. A brief precis of the envisaged future potential of the work is suitable here, but avoid addressing the Nobel Committee directly.}

The implementation of the Riemann Pump in a system design has been presented.
The design process from a charge pump to a custom \gls{ab:dac} has been described.
In contrast to classical current steering topologies this concept used the benefit of differential coding, which saves energy and exhibits a better performance.
A concise evaluation states that the Riemann Pump is a great improvement for conventional digital-to-analog conversion concepts.\\
The major problem was that the current sources could not be defined as necessary, since the transistors have to operate in their saturation region which could be established for all signals.
One requirement would be, that the output voltage only vary between $V_{ss} + V_{ds,sat}$ and $V_{dd} - V_{ds,sat}$.
As simulation showed, this problem occurred for both the high and the low side switching transistor and can be seen that the relative slopes do not scale with the presented slopes.
Non ideal switching occurs which made it necessary to average the current over time.
The problem of not perfect switching is, that the channel is opened and closed slowly in comparison to an ideal switch since the gate of the switching transistor has to be charged which take impact on the driver circuit.
A leakage current over the driver circuit is observed which further increase the problem.
The different behaviour for the switching process of the separate push-pull stages led to a different set of slopes. \\
As no complementary transistors were available in III-V technology, a proper driver circuit had to be investigated.
Implementation of a known concept \cite{MaksimovicPaper} for the driver circuit.
The dimension of the used components were calibrated with respect to the resulting voltage step at an output capacitance which represent the input stage of a power amplifier.
The dimension of the used components was a bottleneck since it limits the signal bandwidth.
Small transistor dimensions could synthesize signals with low frequencies and big dimension with high frequencies.
Therefore the signal bandwidth was investigated to be roughly \SI{1.5}{\GHz} to \SI{6}{\GHz}.\\
Different wave forms could be synthesized (sine wave, rectified sine, triangular).
With an \gls{ab:osr} of four \gls{sy:fsampling} is eight times \gls{sy:fsignal} which has to fit to the transistors unity gain frequency.
For an increase of the \gls{ab:osr} the signal quality is better but the power consumption also linearly scale with it.
The \gls{ab:osr} is limited to the fact that two successive samples have to be distinguished.
A sampling interval which is too small can lead to a non detectable voltage step.\\
Parasitic and loss effects distorted the waveforms.
The highest \gls{ab:snr} (\SI{32.5}{\decibel}) found was for a sine wave at \SI{6}{\giga \hertz} with an amplitude of $\hat{v} = 1.75 V$, an DC-offset of $V_{DC} = 13 V$ and a phase shift of $\phi = -\pi/8$, see appendix \ref{app:snr}.
The system is stable.
The stability check was needed to validate that the circuit did not oscillate.
The energy consumption is critical and is rather in the range for base stations than for mobile devices.
For an increase in signal quality, the resolution could be increased but the whole circuit would become more complex and the energy consumption would increase.
Shifting the baseband to even higher frequencies is possible but the bandwidth stays constant while the power consumption is increased.
The designed test circuit is a proof of concept.
Nevertheless the simulation results confirmed the feasibility of generating different signals.\\
In the realisation and layout process many things had to be considered.
Important for the design of the input lines were that they are of the same length, due to timing issues, as for the bond wires of the in- and output.
The input timing is crucial due to the fact that the switches have to switch synchronous.
One of the most important and crucial things was the dissipation of heat.
Two concepts were presented to dissipate the heat in the most proper way which were fabricated by CONTAG AG.
An improvement for the layout, could be to solder the chips on an electrical insulator while thermal good conductor, as AluminiumNitrid (AlN: 180-200 W/mK -> datasheet).\\
In a first measurement it was shown that the designed circuit converts a digital signal to an analog one.
The measurement follows the expectations of the simulation.
Aside from some parasitic effects the proof of the concept was successful.\\
The design and processing of a \gls{ab:mmic} structure containing the Riemann Pump was beyond the scope of this thesis.
The calculation of the Riemann Code have to be done with an external signal processor, which has to compute this code in real time. 
With the ability to change the reference current and the oversampling ratio it is possible to create arbitrary waveforms.\\
The next steps would be to implement an algorithm which computes the Riemann Code.
A MATLAB algorithm would compute this code by minimizing the deviation 
between a theoretical signal and the synthesized signal.
Then the used chips could be optimized regarding an efficient driver circuit.
This would be to process a \gls{ab:mmic} which would reduce the leakage effects.
Further the switching transistors had to operate in their saturation region, which could be implemented using some kind of diodes.
After optimizing the circuits aspects, some assembly aspect have to be considered.
The implementation of a thermal conductor but electrical insulator for the high side switches would be a huge improvement regarding the heat dissipation.


