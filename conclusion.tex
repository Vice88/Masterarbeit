\chapter{Conclusions and outlook}
\textit{A summary of the most important results, whereby a repeated emphasis of their relevance, importance and novelty cannot hurt. A brief precis of the envisaged future potential of the work is suitable here, but avoid addressing the Nobel Committee directly.}
The calculation of the Riemann Code have to be done with an external signal processor, which has to compute this code in real time. 
This could be a problem, since the energy consumption could increase and the real time calculation.
%Nonetheless the number of signals which can be synthesized with this particular concept is increased with the possibility to change to another combination of slopes to approximate the signal.
%All this calculation should be done with a signal processor and an algorithm.
%With the variation of the slopes and the variation of the sampling time in theory it is possible to create every single signal with more or less good \gls{ab:sqnr}.
 In a more enhanced project a MATLAB algorithm would compute this code by minimizing the deviation between a theoretical signal and the synthesized signal.