\textbf{Agenda}
\begin{enumerate}
	\item literature survey
	\item adaption of push-pull concept from Maksimovic (Talk in June2015)
	\item GaN25 parameter simulation [S-parameter,ON/OFF switching voltage]
	\item determine load impedance [input of PPA - GaN25]
	\item determine dimension of transistors
	\item tuning schematic parameter for optimal simulation (special freq?)
	\item enhancement/extension of 1-bit push-pull to 3-bit push-pull stage
	\item digital input control voltage
	\item determine eight slopes of the current sources in schematic 3-bit resolution
	\item Riemanncode generation with MatLab; minimizing error
	\item control schematic with theoretical input [Riemanncode]
\end{enumerate}
\vspace{1cm}
\textbf{Problems}
\begin{enumerate}
	\item frequency dependent load impedance
	\item absence of p-type transistor makes it hard to efficiently switch the high side transistor in the Gbps range
	\item the heat spreading on the chip and substrate is critical
	\item energy consumption may be very high (mainly switching losses)
	\item the absence of accurate current sources makes it very hard to get a defined slope for the switching transistors.
	\item theoretical slope generation very inaccurate
	\item theoretical slope generation via shorted load  (R = \SI{1}{\ohm})
	\item \textit {$\rightarrow$ slopes ambiguous}?	
	\item \textit{$\rightarrow$ riemanncode generation not possible}?                                                                                                                                                                                                                                                                                                                                                                                                                                                                                                                                                                                                                                                                                                                                                                                                                                                                                                                                                                                                                                                                                                                                                                                                                                                                                                                                                                                                                                             
\end{enumerate}
\vspace{1cm}
\textbf{questions}
\begin{enumerate}
	\item mmW band much higher BW,Datarate,Spectrum - why use the old fashioned frequency bands from DC to 6GHz instead of using a couple of GHz?
	\begin{itemize}
		\item Signal generation is done for the bandwidth of 0..\SI{6}{\GHz}, after that it could be mixed up to higher frequency bands like \SI{47}{\GHz} to \SI{53}{\GHz}
	\end{itemize}
	\item trade off between BW and losses
	\begin{itemize}
		\item higher bandwidth means higher switching speed means higher losses due to the fact that the losses increase linear with the switching speed
	\item higher frequencies means higher attenuation (e.g. weather condition, like rain)
	\end{itemize}
\end{enumerate}