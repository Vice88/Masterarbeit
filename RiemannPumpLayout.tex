\chapter{Layout}
Design of the rogers substrate. Filter network, chip placement, bond, input output connectors everything.
\begin{itemize}
	\item bypass cap dimension
	\begin{itemize}
		\item large package - more inductance - lower freq
		\item higher ESR - bad quality factor - flatten mag of imp vs. freq - broadband good
		\item temp range, voltage range, tolerance, 
		\item dimension: cargo cult principle - rule of thumbs
	\end{itemize}
	\item DC blocking, filter caps (not used) 
	\item bias tees to add bias 
	\item no dc input line because of the bias tee
	\item metal pad size of chip
	\item thermal conduction, waerme abfuehren
	\item line distance
	\item line width, copper height, substrate height, determine the impedance of the msl
	\item no qfn package because the heat would not be dissipated
	\item input line - 50 ohm lines
	\item mmic caps near to the supply pin
	\item equal distance of bonds
	\item bond diameter?
	\item via holes for thermal conduction
	\item backside of chips are metallized
	\item equal length of input lines
	\item 50 ohm output line
	\item metal pad size of chip vs. distance to another pad with vias
	\item coupling could be a problem
	\item sma connector to attach measurement devices
	\item DC power supply with -5V means that ground is 5V
	\item two different layouts
	\begin{itemize}
		\item chip with gnd vias on island and nearby copper plate with thermal vias to cool down the ambient temp of the mmic
		\item chip without gnd via, direct soldered on the copper with thermal vias
	\end{itemize}
\end{itemize}