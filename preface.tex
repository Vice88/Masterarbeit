\chapter{Preface}
Mobile communication became a major part of our daily life. With the release of the fourth mobile communication standard LTE, over seventy 70 'Kraftwerke' (-> EPCOS Ordner gucken !! WICHTIG)
In our every day life applications such as Instagramm, Whatsapp, facebook and Snapchat  are dealing with very high data transfer rates. The industry also handles a very big amount of data. Real time trading at a stock exchange market is crucial, so the industry tries to reach this with the help of RF mobile communication. The data rate is increasing exponentially up to the year 2020. Todays hardware architectures can not handle this amount of data. In the next generation, the fifth, of mobile communication different concepts are needed to deal with this high data rate. In the next generation new hardware architecture are needed. This new concepts are based on the idea of a full software radio. The concept is basically to bring the digital domain as close as possible to the RF Front-End. Therefore the filter, mixer and computation would be much faster, more accurate and less complex. \\
In Chapter two some fundamentals are explain to get a better understanding of the work. Chapter three explains the design workflow to get to an working principle and a schematic. Chapter four evaluates the principle and after a successful simulation the layout is done in chapter five. after designing and layouting the schematic lastly the measurements are taken. in the end the results are discussed.