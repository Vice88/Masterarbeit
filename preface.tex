\chapter{Preface}
Mobile communication became a major part of our daily life. With the release of the fourth mobile communication standard LTE, over seventy 70 'Kraftwerke' (-> EPCOS Ordner gucken !! WICHTIG)
In our every day life applications such as Instagramm, Whatsapp, facebook and Snapchat  are dealing with very high data transfer rates. The industry also handles a very big amount of data. Real time trading at a stock exchange market is crucial, so the industry tries to reach this with the help of RF mobile communication. The data rate is increasing exponentially up to the year 2020. Todays hardware architectures can not handle this amount of data. In the next generation, the fifth, of mobile communication different concepts are needed to deal with this high data rate. In the next generation new hardware architecture are needed. This new concepts are based on the idea of a full software radio. The concept is basically to bring the digital domain as close as possible to the RF Front-End. Therefore the filter, mixer and computation would be much faster, more accurate and less complex. \\
In Chapter two some fundamentals are explain to get a better understanding of the work. Chapter three explains the design workflow to get to an working principle and a schematic. Chapter four evaluates the principle and after a successful simulation the layout is done in chapter five. after designing and layouting the schematic lastly the measurements are taken. in the end the results are discussed.
\chapter*{science-to-go Ambacher}
Sendeleistung der Basisstation betraegt 20W. Elektronische Komponenten brauchen aber mehrere tausend Watt (kW) um die Informationen an den Empfänger zu senden. PA am wichtigsten, hohe leistung, geringe Leistungsaufnahme, hohe frequenz. mehr als 70.000 Basisstationen deutschlandweit, Energieverbrauch/Jahr: 2 Mrd kWh entspricht der jährlich eingespeisten Energie eines kleinen Kohlekraftwerks. Energiebedarf weltweit werden etwa 70 Kernkraftwerke noetig. Mobilfunknutzer steigt: 2020 4.6 Mrd Nutzer, 2020 1800 Billarden Bytes, technisch energieeffiziente loesungen noetig ohne umwelt zu belasten. 5G bis 2020. 1 Mrd bit/s 10mal soviel wie LTE. Extrem schnelle und energieeffiziente power amplifier. Avlanche breakdown! UMTS Basisstation: 20km, LTE mehr Daten weniger Reichweite, hoehere Frequenz: 5km,Reichweite muss in der naechsten Generation auch erhalten bleiben, also viel Leistung auf noch hoeheren Frequenzen. Silizium schafft die Leistung nicht, das Silizium wuerde viel zu heiss, deswegen III-V Verbindungshalbleiter. 3000 Watt Energieaufnahme um mit fuenf antennen jeweils 20 Watt im Umkreis von 20km fuer 600 Telefonate gleichzeitig zu verteilen. Filme, Musik, Stream -> Datenrate und zwar 10 bis 100-fach hoehere Datenrate als LTE. 100 Milliarden Geraete sollen gleichzeitig ansprechbar sein WELTWEIT (Computer,PDA, Auto, Smartphone etc.). Latenzzeiten von unter 1 ms!! Fast Echtzeit, Maschinenkommunikation !! Maschinen sind sehr empfindlich und muessen zu jedem Zeitpunkt wissen wo sie stehen. Energieverbrauch soll um ein tausendstel pro bit gesenkt werden. (insgesamt soll der stromverbrauch um 90 prozent verringert werden) GaAs wird vollstaendig verschwinden, sowohl mobil als auch basisstationen werden auf GaN umruesten muessen. Neue Geraete werden notwendig, 5G wird nicht kompatibel sein mit den alten Standards. 